{\newpage
\phantomsection
\addcontentsline{toc}{chapter}{Bibliography}
\begin{thebibliography}{9}
 
\bibitem{robinson} 
John Alan Robinson.  
\textit{A Machine-oriented logic based on the resolution principle.} 
Journal of the association for computing machinery, Vol. 12 No. 1, January 1965, Pages: 23-41

\bibitem{rob}
John Alan Robinson.  
\textit{Automatic deduction with hyperresolution} 
Int. J. Comp. Math. , Vol. 1, 1965, Pages: 227-234
 
\bibitem{robinson-general} 
John Alan Robinson.  
\textit{The generalized resolution principle.} In: Siekmann J.H., Wrightson G. (eds) Automation of Reasoning. Symbolic Computation (Artificial Intelligence). Springer, Berlin, Heidelberg 1983 Pages: 135-151		

\bibitem{gallier} Jean H. Gallier. \textit{Logic for Computer Science, Foundations of automatic theorem proving}, Dover publications, New York 2003

\bibitem{lifschitz} 
Vladimir Lifschitz.
\textit{Semantical completeness theorems in logic and algebra.} 
American Mathematical Society, Vol. 79 No. 1, May 1980, Pages: 89-96 

\bibitem{Hilbert}
David Hilbert. \textit{\"Uber die vollen Invariantensysteme.} Math. Ann. 42, September 1892, Pages: 313-373

\bibitem{Lang}
Serge Lang, \textit{Algebra}, Springer Graduate Texts in Mathematics, New York 2002


\bibitem{Rabin}
J.L. Rabinowitsch, \textit{Zum Hilbertschen Nullstellensatz} Math. Ann. , 102 (1929) Page: 520
 
\bibitem{mat-2}
Jurij Vladimirovi\v{c} Matijasevi\v{c}.
\textit{Application of the methods of the theory of logical derivation to graph theory.} Mathematicheskie Zametki, Vol. 12 No. 6, December 1972, Pages: 781-790

\bibitem{mat-3}
Jurij Vladimirovi\v{c} Matijasevi\v{c}.
\textit{One scheme for proofs in discrete mathematics} Studies in constructive mathematics and mathematical logic. Part VI, Zap. Nauchn. Sem. LOMI, 40, "Nauka", Leningrad. Otdel., Leningrad, 1974, 94-100 

\bibitem{mat-1} 
Jurij Vladimirovi\v{c} Matijasevi\v{c}.
\textit{Mathematical approach to proving theorems of discrete mathematics.} 
Seminar in mathematics, Steklov mathematical institute, Leningrad, 1975, Pages: 31-48

\bibitem{diestel} Reinhard Diestel. \textit{Graph Theory}, Springer Graduate Texts in Mathematics, Heidelberg 2012



\bibitem{chrom} Gary Chartrand, Ping Zhang. \textit{Chromatic Graph Theory}, Chapman and Hall/CRC, New York 2008

\bibitem{vitaver} L.M. Vitaver \textit{Determination of minimal colorings of graph vertices with the aid of boolean powers of the adjacency matrix.} Dokl. Akad. Nauk SSSR, 147 No 4., 1962, Pages: 758-759

\bibitem{hasse} Maria Hasse \textit{Zur algebraischen Begr\"undung der Graphentheorie. I} Mathematische Nachrichten, 1965, Pages: 275-290

\bibitem{roy}B. Roy, Nombre chromatique et plus longs chemins d'un graph. Rev AFIRO
1 (1967) 127-132.

\newpage

\bibitem{gallai}T. Gallai, \textit{On directed paths and circuits}. In: Theory of Graphs; Proceedings of the Colloquium held at Tihany, Hungary, 1969 (P. Erd ̋s and G. Katona,eds). Academic Press, New York (1969) 115-118.

\bibitem{split} H. Fleischner, \textit{Eulerian Graphs and Related Topics, vol. 1, part 1}, North Holland, 1990.

\bibitem{hadwiger} Hugo Hadwiger, \textit{\"Ungeloste Probleme 26}, Elem. Math. 13 (1958), 128-129.

\bibitem{erdos} B. Bollobás, P. A. Catlin, P.  Erd\"os , \textit{Hadwiger's conjecture is true for almost every graph}, European Journal of Combinatorics (1980), 1: 195-199

\bibitem{fuchs}  L\'aszl\'o Fuchs \textit{Partially ordered algebraic systems}, Pergmon press, 1963

\end{thebibliography}
}